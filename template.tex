%%%%%%%%%%%%%%%%%%%%%%%%%%%%%%%%%%%%%%%%%
% Medium Length Professional CV
% LaTeX Template
% Version 3.0 (December 17, 2022)
%
% This template originates from:
% https://www.LaTeXTemplates.com
%
% Author:
% Vel (vel@latextemplates.com)
%
% Original author:
% Trey Hunner (http://www.treyhunner.com/)
%
% License:
% CC BY-NC-SA 4.0 (https://creativecommons.org/licenses/by-nc-sa/4.0/)
%
%%%%%%%%%%%%%%%%%%%%%%%%%%%%%%%%%%%%%%%%%

%----------------------------------------------------------------------------------------
%	PACKAGES AND OTHER DOCUMENT CONFIGURATIONS
%----------------------------------------------------------------------------------------

\documentclass[
	%a4paper, % Uncomment for A4 paper size (default is US letter)
	11pt, % Default font size, can use 10pt, 11pt or 12pt
]{resume} % Use the resume class

\usepackage{ebgaramond} % Use the EB Garamond font

\usepackage{hyperref}
\hypersetup{
    colorlinks=true,
    linkcolor=blue,
    filecolor=magenta,      
    urlcolor=cyan,
    pdftitle={Overleaf Example},
    pdfpagemode=FullScreen,
    }

%------------------------------------------------

\name{Kumar Abhijeet} % Your name to appear at the top

% You can use the \address command up to 3 times for 3 different addresses or pieces of contact information
% Any new lines (\\) you use in the \address commands will be converted to symbols, so each address will appear as a single line.

\address{Bengaluru \\ India} % Main address

\address{\href{https://www.kumarabhijeet.me/}{portfolio\textbf{X}blog} \\ \href{https://github.com/kumar1202}{Github} \\ \href{https://www.linkedin.com/in/kumar-abhijeet-3b4b4a137/}{LinkedIn}}

\address{(+91)~$\cdot$~8210330554 \\ kumarabhijeet1202@gmail.com} % Contact information

%----------------------------------------------------------------------------------------

\begin{document}

\begin{rSection}{Experience}

        \begin{rSubsection}{\href{https://www.pinelabs.com/}{Pine Labs}}{April 2024 - Present}{Senior Software Engineer - Infra Platforms}{Bengaluru, KA}
            \item Architected the APIOps Platform to manage Kong API Gateways with service equivalence in a multi-regional datacenter setup reducing the~\textbf{API/microservice onboarding time from 1 week to 30 minutes}.
            \item Improved the Kafka Fleet's reliability by~\textbf{three 9s} to improve async messaging among microservices across the org with multi-region deployments, disaster recovery strategies and enhanced observability.
	\end{rSubsection}


	\begin{rSubsection}{\href{https://www.gotocompany.com/en/about-us}{GoTo Group (Gojek)}}{October 2021 - April 2024}{Software Engineer II - Cloud Platform}{Bengaluru, KA}
		\item Established the in-house Database-as-a-Service(DBaaS) ecosystem responsible for handling 100+ datastores across the organisation like Patroni, Postgres, Consul \& Redis leading to the reduction in cloud costs by~\textbf{\$800/datastore/month}.
		\item Designed Terraform modules and Chef cookbooks to provision and configure the above datastores in order to support upto \textbf{5 9s availability} and architected disaster recovery systems to ensure less than \textbf{10MBs in data loss}.

        \item Improved visibility and minimized developer intervention with out-of-the-box alerting, observability and logging using Grafana/Prometheus and ELK stack reducing the \textbf{MTTD to less than 10 minutes}.
          
          \item Decreased end-to-end datastore onboarding time from couple of days to \textbf{just 15 minutes} by utilizing IaC(Infrastructure as Code) abstracted over a Web Interface written in React plugged with a pull request automation using Atlantis.
	\end{rSubsection}

%------------------------------------------------

	
 \begin{rSubsection}{\href{https://www.gotocompany.com/en/about-us}{GoTo Group (Gojek)}}{October 2021 - April 2024}{Software Engineer II - Cloud Platform}{Bengaluru, KA}
	\item Developed the application inventory and orchestration system in Rails to offer Platform-as-a-Service(PaaS) which in turn reduced the application lifecycle period from \textbf{weeks to 30 minutes} for product teams.
		\item Increased dev productivity by saving \textbf{1 hour/day} by incorporating features like continuous deployment, infra provisioning. out-of-the-box logging and observability support into the platform.
          \item Streamlined the user experience to raise the platform's \textbf{MAU by 150\%} with developing a CLI using Ruby's Thor library and Web Portal made with React.
		\item Successfully onboarded and managed \textbf{over 200 microservices}, establishing the internal PaaS as the exclusive developer-facing platform within the organization.
\end{rSubsection}	

%------------------------------------------------

	\begin{rSubsection}{\href{https://www.paypal.com/in/home}{Paypal}}{April 2019 - July 2019}{Software Development Intern - Merchant Analytics}{Bengaluru, KA}
		\item Enhanced the \textbf{customer feedback loop} by implementing a Spring Batch job that extracts and processes merchant feedback data from an ElasticSearch click-stream index along with sending an email to the respective stakeholders.
          \item Developed a Javascript module to automate the generation of release notes from commit messages and tags.
	\end{rSubsection}

\end{rSection}

%----------------------------------------------------------------------------------------
%	EDUCATION SECTION
%----------------------------------------------------------------------------------------

\begin{rSection}{Evangelism}
\textbf{Projects}

	\item \href{https://github.com/kumar1202/gojira}{\textbf{Gojira}} - An opinionated APIOps framework along with a CI/CD tool to enable service/route equivalence across Kong clusters in multiple regions and perform tag based service segregation for case-specific Kong deployments.

\hfill \break.
 
 \textbf{Talks}
 
\item \href{https://speakerdeck.com/kumar_abhijeet/multi-region-apiops-with-kong}{\textbf{Multi-Region APIOps with Kong}} \hfill \textit{Platform Engineering Meetup, BLR} \\ 
	Building APIOps platforms for Kong clusters in a multi-regional datacenter scenario involving service equivalence and data protection constraints.

\item \href{https://youtu.be/m3RpZLaq7H4?si=4n0E7gAYt-GP0pHL}{\textbf{How Patroni solved Database Reliability at Gojek}} \hfill \textit{BangPypers Meetup, BLR} \\ 
	Tackling database reliability at country-scale with Patroni and achieving business value with internal Database-as-a-service Platform.

 \item \href{https://youtu.be/6Eap81aTUaA?si=Q5X0WnvMJGHPcZlm}{\textbf{Crafting Chef Recipes for Reliable Infrastructure}} \hfill \textit{Ruby Users Group, BLR} \\ 
	Best practices for writing Chef recipes for configuring Infra at scale and common pitfalls and deployment patterns with documented experiences.

 

\end{rSection}

\begin{rSection}{Education}
	
	\textbf{RV College of Engineering, Bengaluru} \hfill \textit{Jul 2020} \\ 
	B.E. in Information Science \& Engineering, Minor in Cloud Computing \& Big Data \smallskip \\
	Cumulative GPA: 8.7
	
\end{rSection}

%----------------------------------------------------------------------------------------
%	WORK EXPERIENCE SECTION
%----------------------------------------------------------------------------------------

%----------------------------------------------------------------------------------------
%	TECHNICAL STRENGTHS SECTION
%----------------------------------------------------------------------------------------

\begin{rSection}{Technical Strengths}

	\begin{tabular}{@{} >{\bfseries}l @{\hspace{6ex}} l @{}}
		Languages \& Frameworks & Ruby, Rails, Python, Bash, Javascript, Golang \\
		Infrastructure Tools & Chef, Terraform, Terragrunt, Ansible, Docker, Kubernetes \\
            Datastores & PostgreSQL, Patroni, Consul, Kafka, Redis \\
            Cloud Platforms/OS & GCP, Linux, Bare Metal \\
            DevOps Principles & CI/CD, Containerization, IAC, Monitoring/Logging, Automated  Testing\\
		Protocols \& Tools & REST, JSON, HCL, Git, Vim, zsh
	\end{tabular}

\end{rSection}

%----------------------------------------------------------------------------------------
%	EXAMPLE SECTION
%----------------------------------------------------------------------------------------

%\begin{rSection}{Section Name}

	%Section content\ldots

%\end{rSection}

%----------------------------------------------------------------------------------------

\end{document}
